\documentclass{beamer}

\usepackage[utf8]{inputenc}
\usepackage[french]{babel}
\usepackage{tabularx}
\usepackage{stix}
\usepackage[table]{xcolor}
\usepackage{svg}

\definecolor{ULMviolet}{rgb}{0.356862745, 0.207843137, 0.509803922}
\definecolor{ULMjaune}{rgb}{0.968627451, 0.749019608, 0.0}

\usetheme{Berlin}
\setbeamercolor{palette primary}{bg=ULMviolet,fg=ULMjaune}
\setbeamercolor{palette secondary}{bg=ULMviolet,fg=ULMjaune}
\setbeamercolor{palette tertiary}{bg=ULMviolet,fg=ULMjaune}
\setbeamercolor{palette quaternary}{bg=ULMviolet,fg=ULMjaune}
\setbeamercolor{structure}{fg=ULMviolet}
\setbeamercolor{section in toc}{fg=ULMviolet}

\setbeamercolor{subsection in head/foot}{bg=ULMviolet,fg=ULMjaune}

\usepackage{minted}
\newmintedfile[pcode]{py}{
	linenos,
	breaklines,
	fontsize=\scriptsize,
	numbersep=5pt,
}
\newcommand{\codeframeline}[4]{
	\begin{frame}{#1}
		\centering
		\begin{minipage}{0.9\textwidth}
			\pcode[firstline=#2, lastline=#3]{#4}
		\end{minipage}
	\end{frame}
}

\title{Microprocesseur RISC-V, horloge}
\author{Aghilas BOUSSAA, Marius CAPELLI, Olivier HENRY, Paul WANG}

\AtBeginSection[]
{
	\begin{frame}
		\frametitle{Sommaire}
		\tableofcontents[currentsection]
	\end{frame}
}

\begin{document}
	\maketitle
	
	\begin{frame}{Sommaire}
		\tableofcontents
	\end{frame}
	
	\section{ISA}
	
	\begin{frame}{RISC V}
		\begin{figure}
			\centering
			\includesvg[scale=0.4]{RISC-V-logo.svg}
			\caption{Logo de RISC-V\footnote{RISC-V Foundation, CC BY-SA 4.0}.}
		\end{figure}
	\end{frame}
	
	\begin{frame}{Formats}
		\centering
		\scalebox{0.75}{
			\begin{tabular}{|c|c|c|c|c|c|l|}
				\hline
				\cellcolor{ULMjaune} $\color{ULMviolet}31:25$ & \cellcolor{ULMjaune} $\color{ULMviolet}24:20$ & \cellcolor{ULMjaune} $\color{ULMviolet}19:15$ & \cellcolor{ULMjaune} $\color{ULMviolet}14:12$ & \cellcolor{ULMjaune} $\color{ULMviolet}11:7$ & \cellcolor{ULMjaune} $\color{ULMviolet}6:0$ & \cellcolor{ULMjaune} \color{ULMviolet} type \\
				\hline
				funct7 & rs2 & rs1 & funct3 & rd & opcode & \cellcolor{ULMjaune} \color{ULMviolet} \texttt R \\
				\hline
				\multicolumn{2}{|c|}{imm[$11:0$]} & rs1 & funct3 & rd & opcode & \cellcolor{ULMjaune} \color{ULMviolet} \texttt I \\
				\hline
				imm[$11:5$] & rs2 & rs1 & funct3 & imm[$4:0$] & opcode & \cellcolor{ULMjaune} \color{ULMviolet} \texttt S \\
				\hline
				imm[$12|10:5$] & rs2 & rs1 & funct3 & imm[$4:1|11$] & opcode & \cellcolor{ULMjaune} \color{ULMviolet} \texttt B \\
				\hline
				\multicolumn{4}{|c|}{imm[$31:12$]} & rd & opcode & \cellcolor{ULMjaune} \color{ULMviolet} \texttt U \\
				\hline
				\multicolumn{4}{|c|}{imm[$20|10:1|11|19:12$]} & rd & opcode & \cellcolor{ULMjaune} \color{ULMviolet} \texttt J \\
				\hline
			\end{tabular}
		}
	\end{frame}
	
	\begin{frame}{Exemples}
		\centering
		\scalebox{0.75}{
			\begin{tabular}{|c|c|c|c|c|c|l|}
				\hline
				\cellcolor{ULMjaune} $\color{ULMviolet}31:25$ & \cellcolor{ULMjaune} $\color{ULMviolet}24:20$ & \cellcolor{ULMjaune} $\color{ULMviolet}19:15$ & \cellcolor{ULMjaune} $\color{ULMviolet}14:12$ & \cellcolor{ULMjaune} $\color{ULMviolet}11:7$ & \cellcolor{ULMjaune} $\color{ULMviolet}6:0$ & \cellcolor{ULMjaune} \color{ULMviolet} type \\
				\hline
				0\_00000 & rs2 & rs1 & \_\_\_ & rd & 011\_011 & \cellcolor{ULMjaune} \color{ULMviolet} ALU \\
				\hline
				\multicolumn{2}{|c|}{imm[$11:0$]} & rs1 & \_\_\_ & rd & 001\_011 & \cellcolor{ULMjaune} \color{ULMviolet} ALU-I \\
				\hline
				imm[$11:5$] & rs2 & rs1 & \_\_\_ & imm[$4:0$] & 0100011 & \cellcolor{ULMjaune} \color{ULMviolet} \texttt Store \\
				\hline
				imm[$12|10:5$] & rs2 & rs1 & \_\_\_ & imm[$4:1|11$] & 1100011 & \cellcolor{ULMjaune} \color{ULMviolet} Branch \\
				\hline
				\multicolumn{4}{|c|}{imm[$31:12$]} & rd & 0110111 & \cellcolor{ULMjaune} \color{ULMviolet} LUI \\
				\hline
				\multicolumn{4}{|c|}{imm[$20|10:1|11|19:12$]} & rd & 1101111 & \cellcolor{ULMjaune} \color{ULMviolet} JAL \\
				\hline
			\end{tabular}
		}
	\end{frame}
	
	\begin{frame}{Instructions supportées}
		\centering
		\begin{tabular}{|l|l|}
			\hline
			\cellcolor{ULMjaune} \color{ULMviolet} Catégorie & \cellcolor{ULMjaune} \color{ULMviolet} Instructions \\
			\hline
			\cellcolor{ULMjaune} \color{ULMviolet} ALU & or, and, add, sub, sll, slt, sltu, xor, srl, sra \\
			\hline
			\cellcolor{ULMjaune} \color{ULMviolet} ALU-I & ori, andi, addi, slli, slti, sltiu, xori, srli, srai \\
			\hline
			\cellcolor{ULMjaune} \color{ULMviolet} Jump & jal, jalr \\
			\hline
			\cellcolor{ULMjaune} \color{ULMviolet} Branch & beq, bne, blt, bge, bltu, bgeu \\
			\hline
			\cellcolor{ULMjaune} \color{ULMviolet} Load/Store & lb, lh, lw, ld, lbu,  lhu, lwu, sb, sh, sw, sd \\
			\hline
			\cellcolor{ULMjaune} \color{ULMviolet} Autre & lui, auipc \\
			\hline
		\end{tabular}
	\end{frame}
	
	\section{ALU}
	
	\codeframeline{Carry-lookahead adder}{27}{39}{../alu.py}
	\codeframeline{Réutilisation des circuits}{98}{114}{../alu.py}
	\codeframeline{Multiplicateur peu intelligent}{4}{17}{../alu.py}
	
	\section{RAM}
	
	\begin{frame}{Organisation}
		\centering
		Mots de 8, 16, 32 ou 64 bits, adressable à l'octet, circulaire.
		
		\medskip
		
		\begin{tabularx}{\textwidth}{|l|X|X|X|X|X|X|X|X|}
			\hline
			\cellcolor{ULMjaune}\color{ULMviolet} RAM & \cellcolor{ULMjaune}$\color{ULMviolet}000$ & \cellcolor{ULMjaune}$\color{ULMviolet}100$ & \cellcolor{ULMjaune}$\color{ULMviolet}010$ & \cellcolor{ULMjaune}$\color{ULMviolet}110$ & \cellcolor{ULMjaune}$\color{ULMviolet}001$ & \cellcolor{ULMjaune}$\color{ULMviolet}101$ & \cellcolor{ULMjaune}$\color{ULMviolet}011$ & \cellcolor{ULMjaune}$\color{ULMviolet}111$ \\
			\hline
			00 & $\rightarrow$ & $\rightarrow$ & $\rightarrow$ & $\rightarrow$ & $\rightarrow$ & $\rightarrow$ & $\rightarrow$ & $\carriagereturn$ \\
			\hline
			10 & $\rightarrow$ & $\rightarrow$ & $\rightarrow$ & $\rightarrow$ & $\rightarrow$ & $\rightarrow$ & $\rightarrow$ & $\carriagereturn$ \\
			\hline
			01 & $\rightarrow$ & $\rightarrow$ & $\rightarrow$ & \cellcolor{ULMviolet} $\color{ULMjaune}\rightarrow$ & \cellcolor{ULMviolet} $\color{ULMjaune}\rightarrow$ & \cellcolor{ULMviolet} $\color{ULMjaune}\rightarrow$ & \cellcolor{ULMviolet} $\color{ULMjaune}\rightarrow$ & \cellcolor{ULMviolet} $\color{ULMjaune}\carriagereturn$ \\
			\hline
			11 & \cellcolor{ULMviolet} $\color{ULMjaune}\rightarrow$ & \cellcolor{ULMviolet} $\color{ULMjaune}\rightarrow$ & \cellcolor{ULMviolet} $\color{ULMjaune}\rightarrow$ & $\rightarrow$ & $\rightarrow$ & $\rightarrow$ & $\rightarrow$ & $\carriagereturn$ \\
			\hline
		\end{tabularx}
		
		\medskip
		
		Exemple : \colorbox{ULMjaune}{\color{ULMviolet}110} 01
	\end{frame}
	
	\codeframeline{Calcul des adresses}{38}{45}{../ram.py}
	\codeframeline{Données à écrire}{67}{77}{../ram.py}
	\codeframeline{Modules}{80}{89}{../ram.py}
	
	\section{Programme}
	
	\begin{frame}{RISC V}
		
	\end{frame}
	
\end{document}