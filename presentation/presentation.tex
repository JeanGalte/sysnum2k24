\documentclass{beamer}

\usepackage[utf8]{inputenc}
\usepackage[french]{babel}
\usepackage{tabularx}
\usepackage{stix}
\usepackage[table]{xcolor}
\usepackage{svg}

\definecolor{ULMviolet}{rgb}{0.356862745, 0.207843137, 0.509803922}
\definecolor{ULMjaune}{rgb}{0.968627451, 0.749019608, 0.0}

\usetheme{Berlin}
\setbeamercolor{palette primary}{bg=ULMviolet,fg=ULMjaune}
\setbeamercolor{palette secondary}{bg=ULMviolet,fg=ULMjaune}
\setbeamercolor{palette tertiary}{bg=ULMviolet,fg=ULMjaune}
\setbeamercolor{palette quaternary}{bg=ULMviolet,fg=ULMjaune}
\setbeamercolor{structure}{fg=ULMviolet}
\setbeamercolor{section in toc}{fg=ULMviolet}

\setbeamercolor{subsection in head/foot}{bg=ULMviolet,fg=ULMjaune}

\usepackage{minted}
\newmintedfile[pcode]{py}{
	linenos,
	breaklines,
	fontsize=\scriptsize,
	numbersep=5pt,
}
\newcommand{\codeframeline}[4]{
	\begin{frame}{#1}
		\centering
		\begin{minipage}{0.9\textwidth}
			\pcode[firstline=#2, lastline=#3]{#4}
		\end{minipage}
	\end{frame}
}

\title{Microprocesseur RISC-V, horloge}
\author{Aghilas BOUSSAA, Marius CAPELLI, Olivier HENRY, Paul WANG}

\AtBeginSection[]
{
	\begin{frame}
		\frametitle{Sommaire}
		\tableofcontents[currentsection]
	\end{frame}
}

\begin{document}
	\maketitle
	
	\begin{frame}{Sommaire}
		\tableofcontents
	\end{frame}
	
	\section{ISA}
	
	\begin{frame}{RISC V}
		\begin{figure}
			\centering
			\includesvg[scale=0.4]{RISC-V-logo.svg}
			\caption{Logo de RISC-V\footnote{RISC-V Foundation, CC BY-SA 4.0}.}
		\end{figure}
		
	\end{frame}
	
	\section{ALU}
	
	\begin{frame}{Additionneur}
		carry-lookahead
	\end{frame}
	
	\section{RAM}
	
	\begin{frame}{Organisation}
		\centering
		Mots de 8, 16, 32 ou 64 bits, adressable à l'octet, circulaire.
		
		\medskip
		
		\begin{tabularx}{\textwidth}{|l|X|X|X|X|X|X|X|X|}
			\hline
			\cellcolor{ULMjaune}\color{ULMviolet} RAM & \cellcolor{ULMjaune}$\color{ULMviolet}000$ & \cellcolor{ULMjaune}$\color{ULMviolet}100$ & \cellcolor{ULMjaune}$\color{ULMviolet}010$ & \cellcolor{ULMjaune}$\color{ULMviolet}110$ & \cellcolor{ULMjaune}$\color{ULMviolet}001$ & \cellcolor{ULMjaune}$\color{ULMviolet}101$ & \cellcolor{ULMjaune}$\color{ULMviolet}011$ & \cellcolor{ULMjaune}$\color{ULMviolet}111$ \\
			\hline
			00 & $\rightarrow$ & $\rightarrow$ & $\rightarrow$ & $\rightarrow$ & $\rightarrow$ & $\rightarrow$ & $\rightarrow$ & $\carriagereturn$ \\
			\hline
			10 & $\rightarrow$ & $\rightarrow$ & $\rightarrow$ & $\rightarrow$ & $\rightarrow$ & $\rightarrow$ & $\rightarrow$ & $\carriagereturn$ \\
			\hline
			01 & $\rightarrow$ & $\rightarrow$ & $\rightarrow$ & \cellcolor{ULMviolet} $\color{ULMjaune}\rightarrow$ & \cellcolor{ULMviolet} $\color{ULMjaune}\rightarrow$ & \cellcolor{ULMviolet} $\color{ULMjaune}\rightarrow$ & \cellcolor{ULMviolet} $\color{ULMjaune}\rightarrow$ & \cellcolor{ULMviolet} $\color{ULMjaune}\carriagereturn$ \\
			\hline
			11 & \cellcolor{ULMviolet} $\color{ULMjaune}\rightarrow$ & \cellcolor{ULMviolet} $\color{ULMjaune}\rightarrow$ & \cellcolor{ULMviolet} $\color{ULMjaune}\rightarrow$ & $\rightarrow$ & $\rightarrow$ & $\rightarrow$ & $\rightarrow$ & $\carriagereturn$ \\
			\hline
		\end{tabularx}
		
		\medskip
		
		Exemple : \colorbox{ULMjaune}{\color{ULMviolet}110} 01
	\end{frame}
	
	\codeframeline{Calcul des adresses}{38}{45}{../ram.py}
	\codeframeline{Données à écrire}{67}{77}{../ram.py}
	\codeframeline{Modules}{80}{89}{../ram.py}
	
	\section{Programme}
	
	\begin{frame}{RISC V}
		
	\end{frame}
	
\end{document}