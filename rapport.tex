\documentclass{article}

\usepackage{stmaryrd}

\author{
  Olivier Henry
  \and
  Paul Wang  
  Aghilas Boussa
  \and
  Marius Capelli
}
\date{}

\title{ Rapport du projet de systèmes numériques}

\begin{document}

\maketitle

\newpage

\section{Présentation du projet}

Il s'agit d'une implémentation de RISCV. Elle est incomplète : il lui manque l'addition 32 bits (ce qui n'est pas grave, puisqu'on peut l'obtenir en sommant les deux nombres sur 64 bits puis en tronquant), et quelques instructions mineures. Le projet embarque un assembleur, ainsi qu'une petite horloge écrite dans cet assembleur. La gestion de la RAM et du décodeur font l'objet d'un fichier à part entière. La RAM est en mots de 8 octets, l'addressage se fait sur des octets, mais la lecture permet de lire directement 64 bits d'un coup. 

\section{Unité arithmético logique}

On a codé l'UAL de la façon suivante : 
\begin{itemize}
    \item On récupère \texttt{funct75} depuis le bus 7, et \texttt{funct30}, \texttt{funct31}, \texttt{funct32} depuis le bus 3. Ces canaux permettent d'indiquer l'opération à effectuer. 
    \item On récupère les deux variables (au sens de carotte) \texttt{a} et \texttt{b} sur lesquelles on veut effectuer le calcul. 
    \item On transforme b suivant que l'opération effectuée s'écrit mieux avec b négatif ou non. On n'a donc pas à écrire de \texttt{sub}, on ajoute simplement \texttt{a} et \texttt{-b}. Idem pour \texttt{slt} et \texttt{sltu}.
    \item On précalcule les valeurs qui sont souvent utiles : \texttt{a | b}, \texttt{a \& b}, \texttt{a \^ b}, \texttt{a + b}
    \item On choisit quoi renvoyer suivant ce qu'il y a dans les bus 
\end{itemize}


\begin{table}[h!]
\centering
\begin{tabular}{|c|c|c|c|}
\hline
\textbf{funct3[0]} & \textbf{funct3[1]} & \textbf{funct3[2]} & \textbf{Calcul} \\
\hline
0 & 0 & 0 & a + b \\
0 & 0 & 1 & a $<<$ b \\
0 & 1 & 0 & a $<$ b \\
0 & 1 & 1 & a $<$ b (non signé) \\
1 & 0 & 0 & a \^{} b \\
1 & 0 & 1 & a $>>$ b (arithmétique) \\
1 & 1 & 0 & a $|$ b \\
1 & 1 & 1 & a \& b \\
\hline
\end{tabular}
\caption{Opérations en fonction du bus 3}
\end{table}



\newpage

\section{Assembleur}

Le projet embarque un petit assembleur. Ce dernier gère les opérations usuelles (\texttt{|}, \texttt{\&}, \texttt{+}, \texttt{-}, \texttt{<<}, \texttt{$\leq$}, \texttt{$\leq$ (non signé)}, \texttt{\^}, \texttt{>>} ), le branching, une RAM et des registres $x_i$ pour $ i \in \llbracket 0, 31 \rrbracket$. 

Tableau de correspondance d'instructions avec leur code assembeur. 

\begin{table}[h!]
\centering
\begin{tabular}{|c|c|}
\hline
\textbf{Instruction} & \textbf{Code assembleur} \\
\hline
lb   & LOAD 0b000 \\
lh   & LOAD 0b001 \\
lw   & LOAD 0b010 \\
ld   & LOAD 0b011 \\
lbu  & LOAD 0b100 \\
lhu  & LOAD 0b101 \\
lwu  & LOAD 0b110 \\
sb   & STORE 0b000 \\
sh   & STORE 0b001 \\
sw   & STORE 0b010 \\
sd   & STORE 0b011 \\
\hline
\end{tabular}
\end{table}

Pour une spécification plus grande sur l'encodage des instructions, on pourra se référer au manuel d'instructions RISCV5. On ajoute également les fonctions suivantes pour traiter la date :

\begin{table}[h!]
\centering
\begin{tabular}{|c|c|}
\hline
\textbf{Instruction} & \textbf{Encodage} \\
\hline
gtk   & \[0\] $\times 20$ ++ \texttt{rd} ++ \[0001011\] \\
sdt   & \[0\] $\times 12$ ++ \texttt{rs1} ++ \texttt{funct3} ++ \[0101011\] \\
gdt   & \[0\] $\times 17$ ++ \texttt{funct3} ++ rd ++ \[1011011\] \\
\hline
\end{tabular}
\end{table}

\texttt{gtk} permet de calculer la parité du nombre de seconde, et est utilisée pour "ticker". \texttt{sdt} permet de changer une date, et \texttt{gdt} d'obtenir une date. L'élément traité dans la date est codé par \texttt{funct[3]} : 

\begin{table}[h!]
\centering
\begin{tabular}{|c|c|}
\hline
\textbf{funct[3]} & \textbf{Donnée} \\
\hline
000   & année \\
001  & mois \\
010  & jour \\
011  & heure \\
100  & minute \\
101  & seconde  \\
\hline
\end{tabular}
\end{table}


% GTCK = get_tick (donne la parité du nombre de seconde) 
% GDAY, GYEAR, etc. permet d'obtenir le jour 



\end{document}
